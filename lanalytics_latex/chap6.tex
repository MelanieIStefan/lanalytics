\chapter{Conclusions}
In this dissertation, the use of a tool for analysing low stake-high frequency quizzes was studied. Two objectives for this tool were proposed. The first was the measurement of the difficulty of the items and the abilities of the students, this with the objective to make more suitable the items according to the student abilities. And the second was to propose an open source tool that eases the administration and the analysis of these high-frequency quizzes. To make open source this project, the R language was selected. Finally, for the interactive dashboard, Shiny was selected over other dashboard visualisation tools like Tableau. 

In particular, for the first objective two packages for the Rasch models were explored. The first was the \textit{ltm} package that contains the 1-parameter logistic model (1PL) that contains a difficulty parameter, the 2-parameter logistic model (2PL) that contains, also, a discrimination parameter and the 3-parameter logistic model (3PL) that besides contains a guessing parameter. The second explored package was the \textit{eRm}, which contains the simple Rasch model that is similar to the 1PL. Also, it contains some useful plots to compare the item difficulty with the student abilities. Finally, for the included plots and the numerical stability, the eRm package was selected.

From the second point, the lanalytics package and the lanalytics dashboard were proposed, the package includes some basic descriptive analysis, while the dashboard includes the analysis of the lanalytics package and some plots of the eRm package. Initially, the software was thought to include quizzes from different input formats, but the variability of the inputs was too diverse that the homologation could be troublesome. In this aspect, the lanalytics package can be improved in a great sense. Right now the software accepts a specific *.csv format and also it accepts R data files. Also, the results can be output in .csv format and R data format. 

A focus group was made to evaluate if the objectives of the project were accomplished. In general, the evaluations for the content of the package and the dashboard were evaluated good, but there is a huge area of improvement on the instructions specification, on the provided examples and the explanation of the importance of each plot. Finally, the users agreed that the plots and the content agree with the general objectives of the project. For future work, a series of improvements in the usability should be done. This with the objective to make more attractive the software and to incentivise its use.